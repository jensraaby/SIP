\documentclass[a4paper]{article} 
\usepackage{geometry,fancyhdr,caption,subcaption,graphicx,psfrag,amsfonts,textcomp,mathtools,amsmath,hyperref} 

% settings courtesy of http://www.tjansson.dk/?p=419
\usepackage{listings}
\usepackage{color}
\usepackage{textcomp}
\definecolor{listinggray}{gray}{0.9}
\definecolor{lbcolor}{rgb}{0.9,0.9,0.9}
\lstset{
	backgroundcolor=\color{lbcolor},
	tabsize=4,
	rulecolor=,
	language=matlab,
        basicstyle=\scriptsize,
        upquote=true,
        aboveskip={1.5\baselineskip},
        columns=fixed,
        showstringspaces=false,
        extendedchars=true,
        breaklines=true,
        prebreak = \raisebox{0ex}[0ex][0ex]{\ensuremath{\hookleftarrow}},
        frame=single,
        showtabs=false,
        showspaces=false,
        showstringspaces=false,
        identifierstyle=\ttfamily,
        keywordstyle=\color[rgb]{0,0,1},
        commentstyle=\color[rgb]{0.133,0.545,0.133},
        stringstyle=\color[rgb]{0.627,0.126,0.941},
}

\title{Mandatory exercise 8 \\
Signal and Image Processing 2012} 
\author{Jens P. Raaby \\
\url{frn617@diku.dk}}

\begin{document} 
\maketitle

\section{Area of a Polygon using Green-Stokes' formula}
%compute the area of a polygon given by vertices P1,...,Pn connected by straight line segments.

First the decomposing to partial differentials:
\[
\frac{\partial b}{\partial x} = \frac{1}{2}
\]
\[
\frac{\partial a}{\partial y} = - \frac{1}{2}
\]
So
$
A = \frac{ -y}{2}
$ and
$
B = \frac{ x}{2}
$.

\[
area = \iint_R \frac{1}{2} - (-\frac{1}{2}) dx dy
\]
\[
 = \int (-\frac{y}{2}) dx + \frac{x}{2} dy = \frac{1}{2} \int x dy - ydx
\]

\[
 = \frac{1}{2} \int (-\frac{y}{2}) dx + \frac{x}{2} dy = \frac{1}{2} \int x dy - ydx
\]
For each point $i$ of a polygon this can be worked out as follows:
\[
= \frac{1}{2} ( (x_{i-1} + x_i)(y_i - y_{i-1}) - (y_{i-1} + y_i)(x_i-x_{i-1}) ) = x_{i-1} y_i - x_i y_{i-1}
\]


In the case of a polygon with $n$ points the area formula can be generalised as
\[
A_R = \frac{1}{2} \sum (x_{i-1} y_i - y_{i-1} x_i)
\]
\section{Fourier description \& decimation}
The implementation of Fourier description and decimation is encapsulated in the function jpr\_fourier\_decimate.m (see appendix \ref{appendix-decimate}). The approach is as follows:


\begin{itemize}

    \item Given an image boundary as a series of coordinates, create a complex number c for each point (x,y):
\[
c = x + i y
\]


    \item Fourier transform the vector s containing all the points as complex numbers

    \item Shift the transform so that the high frequencies are at the extremities and the DC component is at the centre.
    
    \item Suppress to zero the coefficients of the shifted transform at the start and end of that vector (according to how many components should be kept)
    
    \item Invert the Fourier shift and transform
    
    \item Convert the resulting vector back to x and y coordinates, and round them to integers
\end{itemize}

I ran the method on a series of leaf silhouettes (downloaded from \url{http://infovisual.info/01/010_en.html}) which I converted to black and white object images using thresholding. They are fairly low resolution images with boundaries with 50-300 points. The reconstructed boundaries of these leaves after decimating 50 components is shown in figure \ref{fig1}. It is clear that the decimation has not completely removed the 
         \begin{figure}
                 \centering
                 \includegraphics[width=\textwidth]{leaves_50}
                 \caption{Leaf objects and the boundaries after decimating to 50 components}
                 \label{fig1}
         \end{figure}


\subsection*{What do you get if you only keep the DC frequency? The DC frequency and the next? the DC and the next two?}
See figure \ref{fig2} for an example where these frequencies are kept on the 4th leaf image from figure \ref{fig1}.
In the case of only keeping the DC frequency, the result is a single point representing the centre of the object. (This is hard to show in a PDF, but the code is also attached in the file ex2.m).

I found it hard to determine what was happening when keeping only the first 2 frequencies, as the resulting boundary seemed to exceed the image area in some cases. The result in any case is a circle which is scaled according to the size of the original shape.

Keeping the DC component and the first 2 higher frequencies gave an elliptical boundary in the same orientation as the original object.

These representations appear to be very much like the lower order moments.
         \begin{figure}
                 \centering
                 \includegraphics[width=\textwidth]{leaf4_first3}
                 \caption{DC,DC+1,DC+2 frequeuncies on 4th leaf image}
                 \label{fig2}
         \end{figure}


DC = centre 
DC+1 = size (circle)
DC+2 = size (2D) 
\clearpage
 \appendix 
\section{jpr\_fourier\_decimate.m source}
\label{appendix-decimate} 
\lstinputlisting{../jpr_fourier_decimate.m}



\end{document} 

